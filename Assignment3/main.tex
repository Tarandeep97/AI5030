\documentclass[journal,12pt,twocolumn]{IEEEtran}

\usepackage{setspace}
\usepackage{gensymb}
\singlespacing
\usepackage[cmex10]{amsmath}
\usepackage{caption}

\usepackage{amsthm}

\usepackage{mathrsfs}
\usepackage{amssymb}
\usepackage{txfonts}
\usepackage{stfloats}
\usepackage{bm}
\usepackage{cite}
\usepackage{cases}
\usepackage{subfig}

\usepackage{longtable}
\usepackage{multirow}

\usepackage{enumitem}
\usepackage{mathtools}
\usepackage{steinmetz}
\usepackage{tikz}
\usepackage{circuitikz}
\usepackage{verbatim}
\usepackage{tfrupee}
\usepackage[breaklinks=true]{hyperref}
\usepackage{graphicx}
\usepackage{tkz-euclide}

\usetikzlibrary{calc,math}
\usepackage{listings}
    \usepackage{color}                                            %%
    \usepackage{array}                                            %%
    \usepackage{longtable}                                        %%
    \usepackage{calc}                                             %%
    \usepackage{multirow}                                         %%
    \usepackage{hhline}                                           %%
    \usepackage{ifthen}                                           %%
    \usepackage{lscape}     
\usepackage{multicol}
\usepackage{chngcntr}

\DeclareMathOperator*{\Res}{Res}

\renewcommand\thesection{\arabic{section}}
\renewcommand\thesubsection{\thesection.\arabic{subsection}}
\renewcommand\thesubsubsection{\thesubsection.\arabic{subsubsection}}

\renewcommand\thesectiondis{\arabic{section}}
\renewcommand\thesubsectiondis{\thesectiondis.\arabic{subsection}}
\renewcommand\thesubsubsectiondis{\thesubsectiondis.\arabic{subsubsection}}


\hyphenation{op-tical net-works semi-conduc-tor}
\def\inputGnumericTable{}                                 %%

\lstset{
%language=C,
frame=single, 
breaklines=true,
columns=fullflexible
}
\begin{document}


\newtheorem{theorem}{Theorem}[section]
\newtheorem{problem}{Problem}
\newtheorem{proposition}{Proposition}[section]
\newtheorem{lemma}{Lemma}[section]
\newtheorem{corollary}[theorem]{Corollary}
\newtheorem{example}{Example}[section]
\newtheorem{definition}[problem]{Definition}

\newcommand{\BEQA}{\begin{eqnarray}}
\newcommand{\EEQA}{\end{eqnarray}}
\newcommand{\define}{\stackrel{\triangle}{=}}
\bibliographystyle{IEEEtran}
\raggedbottom
\setlength{\parindent}{0pt}
\providecommand{\mbf}{\mathbf}
\providecommand{\pr}[1]{\ensuremath{\Pr\left(#1\right)}}
\providecommand{\qfunc}[1]{\ensuremath{Q\left(#1\right)}}
\providecommand{\sbrak}[1]{\ensuremath{{}\left[#1\right]}}
\providecommand{\lsbrak}[1]{\ensuremath{{}\left[#1\right.}}
\providecommand{\rsbrak}[1]{\ensuremath{{}\left.#1\right]}}
\providecommand{\brak}[1]{\ensuremath{\left(#1\right)}}
\providecommand{\lbrak}[1]{\ensuremath{\left(#1\right.}}
\providecommand{\rbrak}[1]{\ensuremath{\left.#1\right)}}
\providecommand{\cbrak}[1]{\ensuremath{\left\{#1\right\}}}
\providecommand{\lcbrak}[1]{\ensuremath{\left\{#1\right.}}
\providecommand{\rcbrak}[1]{\ensuremath{\left.#1\right\}}}
\theoremstyle{remark}
\newtheorem{rem}{Remark}
\newcommand{\sgn}{\mathop{\mathrm{sgn}}}
\providecommand{\abs}[1]{\left\vert#1\right\vert}
\providecommand{\res}[1]{\Res\displaylimits_{#1}} 
\providecommand{\norm}[1]{\left\lVert#1\right\rVert}
%\providecommand{\norm}[1]{\lVert#1\rVert}
\providecommand{\mtx}[1]{\mathbf{#1}}
\providecommand{\mean}[1]{E\left[ #1 \right]}
\providecommand{\fourier}{\overset{\mathcal{F}}{ \rightleftharpoons}}
%\providecommand{\hilbert}{\overset{\mathcal{H}}{ \rightleftharpoons}}
\providecommand{\system}{\overset{\mathcal{H}}{ \longleftrightarrow}}
 %\newcommand{\solution}[2]{\textbf{Solution:}{#1}}
\newcommand{\solution}{\noindent \textbf{Solution: }}
\newcommand{\cosec}{\,\text{cosec}\,}
\providecommand{\dec}[2]{\ensuremath{\overset{#1}{\underset{#2}{\gtrless}}}}
\newcommand{\myvec}[1]{\ensuremath{\begin{pmatrix}#1\end{pmatrix}}}
\newcommand{\mydet}[1]{\ensuremath{\begin{vmatrix}#1\end{vmatrix}}}
\numberwithin{equation}{subsection}
\makeatletter
\@addtoreset{figure}{problem}
\makeatother
\let\StandardTheFigure\thefigure
\let\vec\mathbf
\renewcommand{\thefigure}{\theproblem}
\def\putbox#1#2#3{\makebox[0in][l]{\makeb
ox[#1][l]{}\raisebox{\baselineskip}[0in][0in]{\raisebox{#2}[0in][0in]{#3}}}}
     \def\rightbox#1{\makebox[0in][r]{#1}}
     \def\centbox#1{\makebox[0in]{#1}}
     \def\topbox#1{\raisebox{-\baselineskip}[0in][0in]{#1}}
     \def\midbox#1{\raisebox{-0.5\baselineskip}[0in][0in]{#1}}
\vspace{3cm}
\title{Assignment 3 - Q49, Dec 2018}
\author{Tarandeep Singh}
\maketitle
\newpage
\bigskip
\renewcommand{\thefigure}{\theenumi}
\renewcommand{\thetable}{\theenumi}
Github Link
\begin{lstlisting}
https://github.com/Tarandeep97/AI5030
\end{lstlisting}
\section{Problem}
(Q49, Dec 2018) Let $X \geq 0$ be a random variable on  $(\Omega,{\mathcal {F}}, P)$ with $\mathbb{E}(X)=1$. Let $A\in{\mathcal {F}}$ be an event with $ 0 < P(A) < 1$. Which of the following defines another probability measure on $(\Omega,{\mathcal {F}})$?
\begin{enumerate}
\item $Q(B) = P(A\cap B)$     $\forall B \in {\mathcal {F}}$
\item $Q(B) = P(A\cup B)$     $\forall B \in {\mathcal {F}}$
\item $Q(B) = \mathbb{E}(XI_{B})$     $\forall B \in {\mathcal {F}}$ 
\item $Q(B) = \begin{cases} 
      P(A/B) & P(B)> 0 \\
      0 & P(B) = 0 
   \end{cases}$

\end{enumerate}

\section{Solution}
\textbf{Properties of Probability measure} \\
A Probability measure on $(\Omega,{\mathcal {F}})$ is a function $P:{\mathcal {F}} \rightarrow [0,1]$ satisfying:
\begin{enumerate}
\item $P(\Omega)=1, P(\phi)=0$
\item If $A_{1},A_{2},A_{3}... \in {\mathcal {F}}$  is a collection of disjoint members in ${\mathcal {F}}$, then  \\
 \[P\left(\bigcup_{i=1}^{\infty} A_{i}\right)= \sum_{i=1}^{\infty} P(A_{i}) \]  
\end{enumerate}
\textbf{Investigating Options}\\
Option 1, if $B= \Omega$, then
\begin{align}
    Q(\Omega) = P(A \cap \Omega) = P(A) \neq 1
\end{align}

Option 2, if $B= \phi$, then
\begin{align}
    Q(\phi) = P(A \cup \phi) = P(A) \neq 0
\end{align}
Option 4, if $B= \Omega$, then
\begin{align}
    Q(\Omega) = P(A/\Omega) = \frac{P(A \cap \Omega)}{P(\Omega)} \\
    = P(A)\neq 1
\end{align}
Option 3, Here the indicator function $I_{B}$ defines a Bernoulli random variable. 
\begin{align}
I_{B} = \begin{cases}
1 & \text{ if $x\in B$},\\
0 & \text{ if $x\notin B$},
\end{cases}
\end{align}
if $B=\phi$, 
\begin{align}
    Q(\phi) = E(XI_{\phi}) 
\end{align}
\begin{align}
     \sum_{x\in\Omega} \sum_{x\in\Omega} x.I_{B}(x)P((X=x).(I_{B}(x)))
\end{align}
Here, $\forall x \in \Omega$, $I_{B}(x)=0$, as $B=\phi$. Hence, 
\begin{align}
    Q(\phi) = 0
\end{align}
if $B=\Omega$, 
\begin{align}
    Q(\Omega) = E(XI_{\Omega})
\end{align}
\begin{align}
    =\sum_{x\in\Omega} \sum_{x\in\Omega} x.I_{B}(x)P((X=x).(I_{B}(x)))
\end{align}
Here, $\forall x \in \Omega$, $I_{B}(x)=1$, as $B=\Omega$. Hence,
\begin{align}
    Q(\Omega) = \mathbb{E}(X)=1
\end{align}
Checking for Countable additivity of Q(property 2),
\begin{align}
    Q\left(\bigcup_{i=1}^{\infty} A_{i}\right) = E(XI_{\bigcup_{i=1}^{\infty} A_{i}\right)})
\end{align}
Using Property of Countable additivity on indicator functions, expression can be re-written as below,
\begin{align}
    =E\left(X\sum_{i=1}^{\infty}I_{A_{i}}\right) =E\left(\sum_{i=1}^{\infty}XI_{A_{i}}\right)
\end{align}
\begin{align}
    =\sum_{i=1}^{\infty}E\left(XI_{A_{i}}\right) 
\end{align}
\begin{align}
    = \sum_{i=1}^{\infty}Q(A_{i})
\end{align}
Option 3 satisfies properties of a probability measure. Hence, it is the correct option.\\



\end{document}
