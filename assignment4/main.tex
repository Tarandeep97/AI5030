\documentclass[journal,12pt,twocolumn]{IEEEtran}

\usepackage{setspace}
\usepackage{gensymb}
\singlespacing
\usepackage[cmex10]{amsmath}
\usepackage{caption}

\usepackage{amsthm}

\usepackage{mathrsfs}
\usepackage{amssymb}
\usepackage{txfonts}
\usepackage{stfloats}
\usepackage{bm}
\usepackage{cite}
\usepackage{cases}
\usepackage{subfig}

\usepackage{longtable}
\usepackage{multirow}

\usepackage{enumitem}
\usepackage{mathtools}
\usepackage{steinmetz}
\usepackage{tikz}
\usepackage{circuitikz}
\usepackage{verbatim}
\usepackage{tfrupee}
\usepackage[breaklinks=true]{hyperref}
\usepackage{graphicx}
\usepackage{tkz-euclide}

\usetikzlibrary{calc,math}
\usepackage{listings}
    \usepackage{color}                                            %%
    \usepackage{array}                                            %%
    \usepackage{longtable}                                        %%
    \usepackage{calc}                                             %%
    \usepackage{multirow}                                         %%
    \usepackage{hhline}                                           %%
    \usepackage{ifthen}                                           %%
    \usepackage{lscape}     
\usepackage{multicol}
\usepackage{chngcntr}

\DeclareMathOperator*{\Res}{Res}

\renewcommand\thesection{\arabic{section}}
\renewcommand\thesubsection{\thesection.\arabic{subsection}}
\renewcommand\thesubsubsection{\thesubsection.\arabic{subsubsection}}

\renewcommand\thesectiondis{\arabic{section}}
\renewcommand\thesubsectiondis{\thesectiondis.\arabic{subsection}}
\renewcommand\thesubsubsectiondis{\thesubsectiondis.\arabic{subsubsection}}


\hyphenation{op-tical net-works semi-conduc-tor}
\def\inputGnumericTable{}                                 %%

\lstset{
%language=C,
frame=single, 
breaklines=true,
columns=fullflexible
}
\begin{document}


\newtheorem{theorem}{Theorem}[section]
\newtheorem{problem}{Problem}
\newtheorem{proposition}{Proposition}[section]
\newtheorem{lemma}{Lemma}[section]
\newtheorem{corollary}[theorem]{Corollary}
\newtheorem{example}{Example}[section]
\newtheorem{definition}[problem]{Definition}

\newcommand{\BEQA}{\begin{eqnarray}}
\newcommand{\EEQA}{\end{eqnarray}}
\newcommand{\define}{\stackrel{\triangle}{=}}
\bibliographystyle{IEEEtran}
\raggedbottom
\setlength{\parindent}{0pt}
\providecommand{\mbf}{\mathbf}
\providecommand{\pr}[1]{\ensuremath{\Pr\left(#1\right)}}
\providecommand{\qfunc}[1]{\ensuremath{Q\left(#1\right)}}
\providecommand{\sbrak}[1]{\ensuremath{{}\left[#1\right]}}
\providecommand{\lsbrak}[1]{\ensuremath{{}\left[#1\right.}}
\providecommand{\rsbrak}[1]{\ensuremath{{}\left.#1\right]}}
\providecommand{\brak}[1]{\ensuremath{\left(#1\right)}}
\providecommand{\lbrak}[1]{\ensuremath{\left(#1\right.}}
\providecommand{\rbrak}[1]{\ensuremath{\left.#1\right)}}
\providecommand{\cbrak}[1]{\ensuremath{\left\{#1\right\}}}
\providecommand{\lcbrak}[1]{\ensuremath{\left\{#1\right.}}
\providecommand{\rcbrak}[1]{\ensuremath{\left.#1\right\}}}
\theoremstyle{remark}
\newtheorem{rem}{Remark}
\newcommand{\sgn}{\mathop{\mathrm{sgn}}}
\providecommand{\abs}[1]{\left\vert#1\right\vert}
\providecommand{\res}[1]{\Res\displaylimits_{#1}} 
\providecommand{\norm}[1]{\left\lVert#1\right\rVert}
%\providecommand{\norm}[1]{\lVert#1\rVert}
\providecommand{\mtx}[1]{\mathbf{#1}}
\providecommand{\mean}[1]{E\left[ #1 \right]}
\providecommand{\fourier}{\overset{\mathcal{F}}{ \rightleftharpoons}}
%\providecommand{\hilbert}{\overset{\mathcal{H}}{ \rightleftharpoons}}
\providecommand{\system}{\overset{\mathcal{H}}{ \longleftrightarrow}}
 %\newcommand{\solution}[2]{\textbf{Solution:}{#1}}
\newcommand{\solution}{\noindent \textbf{Solution: }}
\newcommand{\cosec}{\,\text{cosec}\,}
\providecommand{\dec}[2]{\ensuremath{\overset{#1}{\underset{#2}{\gtrless}}}}
\newcommand{\myvec}[1]{\ensuremath{\begin{pmatrix}#1\end{pmatrix}}}
\newcommand{\mydet}[1]{\ensuremath{\begin{vmatrix}#1\end{vmatrix}}}
\numberwithin{equation}{subsection}
\makeatletter
\@addtoreset{figure}{problem}
\makeatother
\let\StandardTheFigure\thefigure
\let\vec\mathbf
\renewcommand{\thefigure}{\theproblem}
\def\putbox#1#2#3{\makebox[0in][l]{\makeb
ox[#1][l]{}\raisebox{\baselineskip}[0in][0in]{\raisebox{#2}[0in][0in]{#3}}}}
     \def\rightbox#1{\makebox[0in][r]{#1}}
     \def\centbox#1{\makebox[0in]{#1}}
     \def\topbox#1{\raisebox{-\baselineskip}[0in][0in]{#1}}
     \def\midbox#1{\raisebox{-0.5\baselineskip}[0in][0in]{#1}}
\vspace{3cm}
\title{Assignment 4 - Q53, June 2018}
\author{Tarandeep Singh}
\maketitle
\newpage
\bigskip
\renewcommand{\thefigure}{\theenumi}
\renewcommand{\thetable}{\theenumi}
Github Link
\begin{lstlisting}
https://github.com/Tarandeep97/AI5030
\end{lstlisting}
\section{Problem}
(Q53, June 2018) Suppose that the lifetime of an electric bulb follows an exponential distribution with mean $\theta$ hours. In order to estimate $\theta$, n bulbs are switched on at the same time. After t hours, $n-m(>0)$ bulbs were found to be in functioning state. If the lifetimes of the other $m(>0)$ bulbs are noted as $x_{1},x_{2},x_{3},...,x_{m}$, respectively, then the maximum likelihood estimate of $\theta$ is given by
\section{Solution}
A continuous random variable $X$ is said to have an exponential distribution with mean $\theta$, $\theta>0$, if its probability density function is given by
\\
\begin{equation*} f(x|\theta) =\left\{ \begin{array}{ll} \frac{1}{\theta} e^{-\frac{1}{\theta} x}, & \hbox{$x\geq0$;} \\ 0 & \hbox{$x<0$} \end{array} \right. \end{equation*}
and its is CDF is given as,
\\
\begin{equation*} F_{X}(x|\theta) =\left\{ \begin{array}{ll} 1- e^{-\frac{1}{\theta} x}, & \hbox{$x\geq0$;} \\ 0 & \hbox{$x<0$} \end{array} \right. \end{equation*}

Lifetime of each m bulb is given as $x_{1},x_{2},..x_{m}$. Then PDF of each of these m bulbs can be written as,
\begin{align}
    f(x_{i}| \theta) = \frac{1}{\theta}e^{-\frac{1}{\theta} x_{i}}
\end{align}
Since, n-m bulbs are still functioning, probability that their lifetime is greater than t, i.e.
\begin{align}
    P(X_{i}>t) = 1 - P(X_{i}\leq t)
\end{align}
\begin{align}
     = 1 - (1- e^{-\frac{1}{\theta} x}) \\
     = e^{-\frac{1}{\theta} x}
\end{align}
The Likelihood function for $\theta$ given data of each bulb is,

\begin{align}
    L(\theta|x_{1},x_{2},..x_{m},t,t..,(n-m)\ times)\\ 
   = \left( \prod_{i=1}^{m} f(x_{i}|\theta) \right)\cdot \left( \prod_{i=1}^{n-m} P(X_{i}>t|\theta) \right)
\end{align}
\begin{align}
    = \frac{1}{\theta^{m}} e^{-\frac{1}{\theta} \sum_{i=1}^{m} x_{i}+(n-m)t}
\end{align}
Using log of the likelihood function,
\begin{align}
    l(\theta|x_{1},x_{2},..x_{m},t,t..,(n-m)\ times) 
\end{align}
\begin{align}
   = \ln(L(\theta|x_{1},x_{2},..x_{m},t,t..,(n-m)\ times))
\end{align}
\begin{align}
   = -m.\ln \theta - \frac{1}{\theta} \left(\sum_{i=1}^{m} x_{i}+(n-m)t\right)
\end{align}
Maxima of the above function gives us MLE of $\theta$, i.e. when
\begin{align}
    \frac{d\left(-m.\ln \theta - \frac{1}{\theta} \left(\sum_{i=1}^{m} x_{i}+(n-m)t\right)\right)}{d\theta} = 0
\end{align}
\begin{align}
    -m\frac{1}{\theta}+\frac{1}{\theta^2}\left(\sum_{i=1}^{m} x_{i}+(n-m)t\right) = 0
\end{align}
\begin{align}
    \hat{\theta} = \frac{\sum_{i=1}^{m} x_{i}+(n-m)t}{m}
\end{align}

\end{document}
