\documentclass[journal,12pt,twocolumn]{IEEEtran}

\usepackage{setspace}
\usepackage{gensymb}
\singlespacing
\usepackage[cmex10]{amsmath}
\usepackage{caption}

\usepackage{amsthm}

\usepackage{mathrsfs}
\usepackage{amsmath}
\usepackage{amssymb}
\usepackage{txfonts}
\usepackage{stfloats}
\usepackage{bm}
\usepackage{cite}
\usepackage{cases}
\usepackage{subfig}

\usepackage{longtable}
\usepackage{multirow}

\usepackage{enumitem}
\usepackage{mathtools}
\usepackage{steinmetz}
\usepackage{tikz}
\usepackage{circuitikz}
\usepackage{verbatim}
\usepackage{tfrupee}
\usepackage[breaklinks=true]{hyperref}
\usepackage{graphicx}
\usepackage{tkz-euclide}

\usetikzlibrary{calc,math}
\usepackage{listings}
    \usepackage{color}                                            %%
    \usepackage{array}                                            %%
    \usepackage{longtable}                                        %%
    \usepackage{calc}                                             %%
    \usepackage{multirow}                                         %%
    \usepackage{hhline}                                           %%
    \usepackage{ifthen}                                           %%
    \usepackage{lscape}     
\usepackage{multicol}
\usepackage{chngcntr}

\DeclareMathOperator*{\Res}{Res}

\renewcommand\thesection{\arabic{section}}
\renewcommand\thesubsection{\thesection.\arabic{subsection}}
\renewcommand\thesubsubsection{\thesubsection.\arabic{subsubsection}}

\renewcommand\thesectiondis{\arabic{section}}
\renewcommand\thesubsectiondis{\thesectiondis.\arabic{subsection}}
\renewcommand\thesubsubsectiondis{\thesubsectiondis.\arabic{subsubsection}}


\hyphenation{op-tical net-works semi-conduc-tor}
\def\inputGnumericTable{}                                 %%

\lstset{
%language=C,
frame=single, 
breaklines=true,
columns=fullflexible
}
\begin{document}


\newtheorem{theorem}{Theorem}[section]
\newtheorem{problem}{Problem}
\newtheorem{proposition}{Proposition}[section]
\newtheorem{lemma}{Lemma}[section]
\newtheorem{corollary}[theorem]{Corollary}
\newtheorem{example}{Example}[section]
\newtheorem{definition}[problem]{Definition}

\newcommand{\BEQA}{\begin{eqnarray}}
\newcommand{\EEQA}{\end{eqnarray}}
\newcommand{\define}{\stackrel{\triangle}{=}}
\bibliographystyle{IEEEtran}
\raggedbottom
\setlength{\parindent}{0pt}
\providecommand{\mbf}{\mathbf}
\providecommand{\pr}[1]{\ensuremath{\Pr\left(#1\right)}}
\providecommand{\qfunc}[1]{\ensuremath{Q\left(#1\right)}}
\providecommand{\sbrak}[1]{\ensuremath{{}\left[#1\right]}}
\providecommand{\lsbrak}[1]{\ensuremath{{}\left[#1\right.}}
\providecommand{\rsbrak}[1]{\ensuremath{{}\left.#1\right]}}
\providecommand{\brak}[1]{\ensuremath{\left(#1\right)}}
\providecommand{\lbrak}[1]{\ensuremath{\left(#1\right.}}
\providecommand{\rbrak}[1]{\ensuremath{\left.#1\right)}}
\providecommand{\cbrak}[1]{\ensuremath{\left\{#1\right\}}}
\providecommand{\lcbrak}[1]{\ensuremath{\left\{#1\right.}}
\providecommand{\rcbrak}[1]{\ensuremath{\left.#1\right\}}}
\theoremstyle{remark}
\newtheorem{rem}{Remark}
\newcommand{\sgn}{\mathop{\mathrm{sgn}}}
\providecommand{\abs}[1]{\left\vert#1\right\vert}
\providecommand{\res}[1]{\Res\displaylimits_{#1}} 
\providecommand{\norm}[1]{\left\lVert#1\right\rVert}
%\providecommand{\norm}[1]{\lVert#1\rVert}
\providecommand{\mtx}[1]{\mathbf{#1}}
\providecommand{\mean}[1]{E\left[ #1 \right]}
\providecommand{\fourier}{\overset{\mathcal{F}}{ \rightleftharpoons}}
%\providecommand{\hilbert}{\overset{\mathcal{H}}{ \rightleftharpoons}}
\providecommand{\system}{\overset{\mathcal{H}}{ \longleftrightarrow}}
 %\newcommand{\solution}[2]{\textbf{Solution:}{#1}}
\newcommand{\solution}{\noindent \textbf{Solution: }}
\newcommand{\cosec}{\,\text{cosec}\,}
\providecommand{\dec}[2]{\ensuremath{\overset{#1}{\underset{#2}{\gtrless}}}}
\newcommand{\myvec}[1]{\ensuremath{\begin{pmatrix}#1\end{pmatrix}}}
\newcommand{\mydet}[1]{\ensuremath{\begin{vmatrix}#1\end{vmatrix}}}
\numberwithin{equation}{subsection}
\makeatletter
\@addtoreset{figure}{problem}
\makeatother
\let\StandardTheFigure\thefigure
\let\vec\mathbf
\renewcommand{\thefigure}{\theproblem}
\def\putbox#1#2#3{\makebox[0in][l]{\makeb
ox[#1][l]{}\raisebox{\baselineskip}[0in][0in]{\raisebox{#2}[0in][0in]{#3}}}}
     \def\rightbox#1{\makebox[0in][r]{#1}}
     \def\centbox#1{\makebox[0in]{#1}}
     \def\topbox#1{\raisebox{-\baselineskip}[0in][0in]{#1}}
     \def\midbox#1{\raisebox{-0.5\baselineskip}[0in][0in]{#1}}
\vspace{3cm}
\title{Assignment 7 - Q7, GATE 2021}
\author{Tarandeep Singh}
\maketitle
\newpage
\bigskip
\renewcommand{\thefigure}{\theenumi}
\renewcommand{\thetable}{\theenumi}
Github Link
\begin{lstlisting}
https://github.com/Tarandeep97/AI5030
\end{lstlisting}
\section{Problem}
(Q7, GATE 2021) Let the joint distribution of (X,Y) be bivariate normal with the mean vector \begin{pmatrix}0\\0\end{pmatrix} and variance-covariance matrix \begin{pmatrix}1\ \rho\\ \rho\ 1\end{pmatrix}
, where -$1<\rho<1$. Let $\phi_{\rho}(0,0) = P(X\leq0,Y\leq0)$. Then the Kendall’s $\tau$ coefficient between X and Y equals
\section{Solution}
Kendall's $\tau$ coefficient between X and Y equals is defined as 
\begin{align}
    P((X-X^\prime)(Y-Y^\prime)>0)-P((X-X^\prime)(Y-Y^\prime)<0)
\end{align}
where $(X^\prime, Y^\prime)$ is bivariate normal independent of X and Y.
\begin{align}
    \begin{pmatrix}X^\prime\\Y^\prime\end{pmatrix} \sim N_{2} \left[\begin{pmatrix}0\\0\end{pmatrix},\begin{pmatrix}1\ \rho\\ \rho\ 1\end{pmatrix} \right]
\end{align}
Using equation 2.0.1,
\begin{align}
    \tau_{\rho}(X,Y)= 2.P((X-X^\prime)(Y-Y^\prime)>0)-1
\end{align}
Let $Z_{1}= X-X^\prime$ and $Z_{2}= Y-Y^\prime$
\begin{align}
    \tau_{\rho}(X,Y)= 2.P(Z_{1}Z_{2}>0)-1
\end{align}
\begin{align}
    = 2.[P(Z_{1}>0,Z_{2}>0)]-1
\end{align}
Expectation of $Z_{1}$ and $Z_{2}$
\begin{align}
   E[Z_{1}] = E[X-X^\prime] = E[X]-E[X^\prime] = 0
\end{align}
\begin{align}
   E[Z_{2}] = E[Y-Y^\prime] = E[Y]-E[Y^\prime] = 0
\end{align}
Variance of $Z_{1}$ and $Z_{2}$
\begin{align}
   Var(Z_{1}) = Var(X)+Var(X^\prime) = 2
\end{align}
\begin{align}
   Var(Z_{2}) = Var(Y)+Var(Y^\prime) = 2
\end{align}
\begin{align}
   Cov(Z_{1},Z_{2}) = Cov (X-X^\prime,Y-Y^\prime)
\end{align}
\begin{align}
  = Cov(X,Y)-Cov(X,Y^\prime)-Cov(X^\prime,Y)+Cov(X^\prime,Y^\prime)
\end{align}
\begin{align}
    Cov(Z_{1},Z_{2}) = 2\rho
\end{align}
So, distribution of bivariate (Z1,Z2) is 
\begin{align}
\begin{pmatrix}Z_{1}\\Z_{2}\end{pmatrix}=\begin{pmatrix}X-X^\prime\\Y-Y^\prime\end{pmatrix} \sim N_{2} \left[\begin{pmatrix}0\\0\end{pmatrix},\begin{pmatrix}2\ 2\rho\\ 2\rho\ 2\end{pmatrix} \right]
\end{align}
This further implies,
\begin{align}
    \frac{1}{\sqrt{2}}\begin{pmatrix}Z_{1}\\Z_{2}\end{pmatrix}=\begin{pmatrix}X\\Y\end{pmatrix}
\end{align}
So, equation 2.0.5 can be rewritten as,
\begin{align}
    \tau_{\rho}(X,Y) = 2.[P(X>0,Y>0)]-1
\end{align}
Now, by symmetry,
\begin{align}
    \tau_{\rho}(X,Y) = 2.2.[P(X\leq0,Y\leq0]]-1
\end{align}
\begin{align}
    \tau_{\rho}(X,Y) = 4.\phi_{\rho}(0,0)-1
\end{align}
\end{document}
