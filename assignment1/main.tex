\documentclass[journal,12pt,twocolumn]{IEEEtran}

\usepackage{setspace}
\usepackage{gensymb}
\singlespacing
\usepackage[cmex10]{amsmath}

\usepackage{amsthm}

\usepackage{mathrsfs}
\usepackage{txfonts}
\usepackage{stfloats}
\usepackage{bm}
\usepackage{cite}
\usepackage{cases}
\usepackage{subfig}

\usepackage{longtable}
\usepackage{multirow}

\usepackage{enumitem}
\usepackage{mathtools}
\usepackage{steinmetz}
\usepackage{tikz}
\usepackage{circuitikz}
\usepackage{verbatim}
\usepackage{tfrupee}
\usepackage[breaklinks=true]{hyperref}
\usepackage{graphicx}
\usepackage{tkz-euclide}

\usetikzlibrary{calc,math}
\usepackage{listings}
    \usepackage{color}                                            %%
    \usepackage{array}                                            %%
    \usepackage{longtable}                                        %%
    \usepackage{calc}                                             %%
    \usepackage{multirow}                                         %%
    \usepackage{hhline}                                           %%
    \usepackage{ifthen}                                           %%
    \usepackage{lscape}     
\usepackage{multicol}
\usepackage{chngcntr}

\DeclareMathOperator*{\Res}{Res}

\renewcommand\thesection{\arabic{section}}
\renewcommand\thesubsection{\thesection.\arabic{subsection}}
\renewcommand\thesubsubsection{\thesubsection.\arabic{subsubsection}}

\renewcommand\thesectiondis{\arabic{section}}
\renewcommand\thesubsectiondis{\thesectiondis.\arabic{subsection}}
\renewcommand\thesubsubsectiondis{\thesubsectiondis.\arabic{subsubsection}}


\hyphenation{op-tical net-works semi-conduc-tor}
\def\inputGnumericTable{}                                 %%

\lstset{
%language=C,
frame=single, 
breaklines=true,
columns=fullflexible
}
\begin{document}


\newtheorem{theorem}{Theorem}[section]
\newtheorem{problem}{Problem}
\newtheorem{proposition}{Proposition}[section]
\newtheorem{lemma}{Lemma}[section]
\newtheorem{corollary}[theorem]{Corollary}
\newtheorem{example}{Example}[section]
\newtheorem{definition}[problem]{Definition}

\newcommand{\BEQA}{\begin{eqnarray}}
\newcommand{\EEQA}{\end{eqnarray}}
\newcommand{\define}{\stackrel{\triangle}{=}}
\bibliographystyle{IEEEtran}
\raggedbottom
\setlength{\parindent}{0pt}
\providecommand{\mbf}{\mathbf}
\providecommand{\pr}[1]{\ensuremath{\Pr\left(#1\right)}}
\providecommand{\qfunc}[1]{\ensuremath{Q\left(#1\right)}}
\providecommand{\sbrak}[1]{\ensuremath{{}\left[#1\right]}}
\providecommand{\lsbrak}[1]{\ensuremath{{}\left[#1\right.}}
\providecommand{\rsbrak}[1]{\ensuremath{{}\left.#1\right]}}
\providecommand{\brak}[1]{\ensuremath{\left(#1\right)}}
\providecommand{\lbrak}[1]{\ensuremath{\left(#1\right.}}
\providecommand{\rbrak}[1]{\ensuremath{\left.#1\right)}}
\providecommand{\cbrak}[1]{\ensuremath{\left\{#1\right\}}}
\providecommand{\lcbrak}[1]{\ensuremath{\left\{#1\right.}}
\providecommand{\rcbrak}[1]{\ensuremath{\left.#1\right\}}}
\theoremstyle{remark}
\newtheorem{rem}{Remark}
\newcommand{\sgn}{\mathop{\mathrm{sgn}}}
\providecommand{\abs}[1]{\left\vert#1\right\vert}
\providecommand{\res}[1]{\Res\displaylimits_{#1}} 
\providecommand{\norm}[1]{\left\lVert#1\right\rVert}
%\providecommand{\norm}[1]{\lVert#1\rVert}
\providecommand{\mtx}[1]{\mathbf{#1}}
\providecommand{\mean}[1]{E\left[ #1 \right]}
\providecommand{\fourier}{\overset{\mathcal{F}}{ \rightleftharpoons}}
%\providecommand{\hilbert}{\overset{\mathcal{H}}{ \rightleftharpoons}}
\providecommand{\system}{\overset{\mathcal{H}}{ \longleftrightarrow}}
 %\newcommand{\solution}[2]{\textbf{Solution:}{#1}}
\newcommand{\solution}{\noindent \textbf{Solution: }}
\newcommand{\cosec}{\,\text{cosec}\,}
\providecommand{\dec}[2]{\ensuremath{\overset{#1}{\underset{#2}{\gtrless}}}}
\newcommand{\myvec}[1]{\ensuremath{\begin{pmatrix}#1\end{pmatrix}}}
\newcommand{\mydet}[1]{\ensuremath{\begin{vmatrix}#1\end{vmatrix}}}
\numberwithin{equation}{subsection}
\makeatletter
\@addtoreset{figure}{problem}
\makeatother
\let\StandardTheFigure\thefigure
\let\vec\mathbf
\renewcommand{\thefigure}{\theproblem}
\def\putbox#1#2#3{\makebox[0in][l]{\makeb
ox[#1][l]{}\raisebox{\baselineskip}[0in][0in]{\raisebox{#2}[0in][0in]{#3}}}}
     \def\rightbox#1{\makebox[0in][r]{#1}}
     \def\centbox#1{\makebox[0in]{#1}}
     \def\topbox#1{\raisebox{-\baselineskip}[0in][0in]{#1}}
     \def\midbox#1{\raisebox{-0.5\baselineskip}[0in][0in]{#1}}
\vspace{3cm}
\title{Assignment 1}
\author{Tarandeep Singh}
\maketitle
\newpage
\bigskip
\renewcommand{\thefigure}{\theenumi}
\renewcommand{\thetable}{\theenumi}
Github Link
\begin{lstlisting}
https://github.com/Tarandeep97/AI5030
\end{lstlisting}
\section{Problem}
(51) Consider a Markov Chain with state space  \{0,1,2,3,4\} and transition matrix
\begin{enumerate}
\begin{align}
P=\begin{matrix} & \begin{matrix}0&&1&&2&&3 && 4\end{matrix} \\\\ \begin{matrix}0\\\\1\\\\2\\\\3\\\\4\end{matrix} & \begin{pmatrix} 1 & 0 & 0 & 0 & 0 \\\\  1 / 3 & 1 / 3 & 1 / 3 & 0 & 0 \\\\  0 & 1 / 3 & 1 / 3 & 1 / 3 & 0 \\\\ 0 & 0 & 1 / 3 & 1 / 3 & 1 / 3 \\\\  0 & 0 & 0 & 0 & 1\end{pmatrix}\\\\ \end{matrix}
\end{align}
\end{enumerate}
Then \lim _{n \rightarrow \infty} p_{23}^{(n)} equals?
\endenumerate}
\section{Solution}
Here $p_{23}^{(n)}$ can be written as follows,
\begin{align}
   p_{23}^{(n)} = P\{X_{n}=2|X_{0}=3\} 
\end{align}
i.e. probability of reaching state 2 after n steps from state 3.
\\
From the given transition matrix,
\begin{align}
   p_{23}^{(1)} = 1/3
\end{align}
This is the initial probability as per transition matrix P. 
\\
In order to find $p_{ij}^{(n)}$, corresponding entry of $P^n$ matrix is required.
To find $P^n$, diagonalized form of P is required.  \\
Using characteristic equation,
\begin{align}
    |P-\lambda I|=0
\end{align}
\begin{align}
    \left|\begin{array}{ccccc}1 - \lambda & 0 & 0 & 0 & 0\\\frac{1}{3} & \frac{1}{3} - \lambda & \frac{1}{3} & 0 & 0\\0 & \frac{1}{3} & \frac{1}{3} - \lambda & \frac{1}{3} & 0\\0 & 0 & \frac{1}{3} & \frac{1}{3} - \lambda & \frac{1}{3}\\0 & 0 & 0 & 0 & 1 - \lambda\end{array}\right| = 0
\end{align}
\begin{align}
\left(1-\lambda\right)^{2} \left(- \lambda^{3} + \lambda^{2} - \frac{\lambda}{9} - \frac{1}{27}\right) = 0
\end{align}
On solving, below are eigen values and corresponding eigen vectors of P
\begin{align}
\lambda_{1} = \frac{1}{3}, \left[\begin{array}{c}0\\-1\\0\\1\\0\end{array}\right]\right\ \\
\lambda_{2} = 1,\lambda_{3} = 1, \left[\begin{array}{c}4\\3\\2\\1\\0\end{array}\right], \left[\begin{array}{c}-3\\-2\\-1\\0\\1\end{array}\right]
\\
\lambda_{4} = 
- \frac{-1 + \sqrt{2}}{3}, \left[\begin{array}{c}0\\1\\- \sqrt{2}\\1\\0\end{array}\right]
\\
\lambda_{5}= 
\frac{1 + \sqrt{2}}{3},\left[\begin{array}{c}0\\1\\\sqrt{2}\\1\\0\end{array}\right]
\end{align}
Diagnolizing P from obtained eigen values and vectors,
\begin{align}
    P = X D X^{-1}
\end{align}
where,
\begin{align}
    D = \left[\begin{array}{ccccc}\frac{1}{3} & 0 & 0 & 0 & 0\\0 & 1 & 0 & 0 & 0\\0 & 0 & 1 & 0 & 0\\0 & 0 & 0 & - \frac{-1 + \sqrt{2}}{3} & 0\\0 & 0 & 0 & 0 & \frac{1 + \sqrt{2}}{3}\end{array}\right]
\end{align}
\begin{align}
    X = \left[\begin{array}{ccccc}0 & 4 & -3 & 0 & 0\\-1 & 3 & -2 & 1 & 1\\0 & 2 & -1 & - \sqrt{2} & \sqrt{2}\\1 & 1 & 0 & 1 & 1\\0 & 0 & 1 & 0 & 0\end{array}\right]
\end{align}
\begin{align}
    X^{-1} = \left[\begin{array}{ccccc}\frac{1}{4} & - \frac{1}{2} & 0 & \frac{1}{2} & - \frac{1}{4}\\\frac{1}{4} & 0 & 0 & 0 & \frac{3}{4}\\0 & 0 & 0 & 0 & 1\\\frac{-2 + \sqrt{2}}{8} & \frac{1}{4} & - \frac{\sqrt{2}}{4} & \frac{1}{4} & \frac{-2 + \sqrt{2}}{8}\\- \frac{\sqrt{2} + 2}{8} & \frac{1}{4} & \frac{\sqrt{2}}{4} & \frac{1}{4} & - \frac{\sqrt{2} + 2}{8}\end{array}\right]
\end{align}
\begin{align}
    P^{n} = X D^{n}X^{-1}
\end{align}
After obtaining $X D^{n}X^{-1}$, the required entry comes out to be
\begin{align}
    P^{n}[3][2] = \frac{\sqrt{2}\left(\left(1 - \sqrt{2}\right)^{n} - \left(1 + \sqrt{2}\right)^{n}\right)}{4\cdot3^{n}} 
\end{align}
As n $\rightarrow\infty$,
\begin{align}
    P^{n}[3][2] = 0
\end{align}
Hence,
  \lim _{n \rightarrow \infty} p_{23}^{(n)} = 0 
\begin{figure}[!ht]
\centering
         \includegraphics[width=\columnwidth,fbox]{NStepProb.PNG}
         \caption{Change in probability wrt n}
         \label{Figure}
\end{figure}
\section{Appendix}
\textbf{Calculation of Eigen Values and Eigen Vectors}
\begin{align}
    \left|\begin{array}{ccccc}1 - \lambda & 0 & 0 & 0 & 0\\\frac{1}{3} & \frac{1}{3} - \lambda & \frac{1}{3} & 0 & 0\\0 & \frac{1}{3} & \frac{1}{3} - \lambda & \frac{1}{3} & 0\\0 & 0 & \frac{1}{3} & \frac{1}{3} - \lambda & \frac{1}{3}\\0 & 0 & 0 & 0 & 1 - \lambda\end{array}\right| = 0
\end{align}
Expanding along row 1:
\begin{align}
(1-\lambda)
 \left|\begin{array}{cccc}\frac{1}{3} - \lambda & \frac{1}{3} & 0 & 0\\\frac{1}{3} & \frac{1}{3} - \lambda & \frac{1}{3} & 0\\0 & \frac{1}{3} & \frac{1}{3} - \lambda & \frac{1}{3}\\0 & 0 & 0 & 1 - \lambda\end{array}\right|
\end{align}
Expanding along row 4:
\begin{align}
(1-\lambda)^2
 \left|\begin{array}{ccc}\frac{1}{3} - \lambda & \frac{1}{3} & 0 \\\frac{1}{3} & \frac{1}{3} - \lambda & \frac{1}{3} \\0 & \frac{1}{3} & \frac{1}{3} - \lambda \end{array}\right| 
\end{align}
Performing $C_{1} = C_{1} - \left(1 - 3 \lambda\right) C_{2}$
\begin{align}
(1-\lambda)^2
 \left|\begin{array}{ccc}0 & \frac{1}{3} & 0\\\lambda \left(2 - 3 \lambda\right) & \frac{1}{3} - \lambda & \frac{1}{3}\\\lambda - \frac{1}{3} & \frac{1}{3} & \frac{1}{3} - \lambda\end{array}\right|
\end{align}
Expanding along row 1:
\begin{align}
(1-\lambda)^2\cdot \left(-\frac{1}{3}\right) 
\left(\left(\lambda \left(2 - 3 \lambda\right)\right)\cdot \left(\frac{1}{3} - \lambda\right) - \left(\frac{1}{3}\right)\cdot \left(\lambda - \frac{1}{3}\right)\right)
\end{align}
\begin{align}
\left(1-\lambda\right)^{2} \left(- \lambda^{3} + \lambda^{2} - \frac{\lambda}{9} - \frac{1}{27}\right) = 0
\end{align}
Solving this equation,
\begin{align}
    \lambda = \frac{1}{3},1,1,- \frac{-1 + \sqrt{2}}{3},\frac{1 + \sqrt{2}}{3}
\end{align}
For  $\lambda = \frac{1}{3}$, we can find eigen vector by putting its values in characterstic equation
\begin{align}
    \left[\begin{array}{ccccc}\frac{2}{3} & 0 & 0 & 0 & 0\\\frac{1}{3} & 0 & \frac{1}{3} & 0 & 0\\0 & \frac{1}{3} & 0 & \frac{1}{3} & 0\\0 & 0 & \frac{1}{3} & 0 & \frac{1}{3}\\0 & 0 & 0 & 0 & \frac{2}{3}\end{array}\right]
\end{align}
Finding row reduced echelon form of matrix and finding solution,
\begin{align}
    \left[\begin{array}{ccccc}1 & 0 & 0 & 0 & 0\\0 & 1 & 0 & 1 & 0\\0 & 0 & 1 & 0 & 0\\0 & 0 & 0 & 0 & 1\\0 & 0 & 0 & 0 & 0\end{array}\right]\left[\begin{array}{c}x_{1}\\x_{2}\\x_{3}\\x_{4}\\x_{5}\end{array}\right] = \left[\begin{array}{c}0\\0\\0\\0\\0\end{array}\right]
\end{align}
Here,
\begin{align}
    x_{2} + x_{4} = 0 \\
    x_{1} =x_{3} = x_{5} = 0 
\end{align}
Let  $x_{2} = -t$ \\
From this, below is the eigen vector,
\begin{align}
    \left[\begin{array}{c}x_{1}\\x_{2}\\x_{3}\\x_{4}\\x_{5}\end{array}\right] = \left[\begin{array}{c}0\\-1\\0\\1\\0\end{array}\right]\right\ t
\end{align}
Similary, we can obtain for other Eigen Values.
\begin{align}
\lambda_{2} = 1,\lambda_{3} = 1, \left[\begin{array}{c}4\\3\\2\\1\\0\end{array}\right], \left[\begin{array}{c}-3\\-2\\-1\\0\\1\end{array}\right]
\\
\lambda_{4} = 
- \frac{-1 + \sqrt{2}}{3}, \left[\begin{array}{c}0\\1\\- \sqrt{2}\\1\\0\end{array}\right]
\\
\lambda_{5}= 
\frac{1 + \sqrt{2}}{3},\left[\begin{array}{c}0\\1\\\sqrt{2}\\1\\0\end{array}\right]
\end{align}
\\ 
\\

\end{document}
