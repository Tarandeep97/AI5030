\documentclass[journal,12pt,twocolumn]{IEEEtran}

\usepackage{setspace}
\usepackage{gensymb}
\singlespacing
\usepackage[cmex10]{amsmath}
\usepackage{caption}

\usepackage{amsthm}

\usepackage{mathrsfs}
\usepackage{amssymb}
\usepackage{txfonts}
\usepackage{stfloats}
\usepackage{bm}
\usepackage{cite}
\usepackage{cases}
\usepackage{subfig}

\usepackage{longtable}
\usepackage{multirow}

\usepackage{enumitem}
\usepackage{mathtools}
\usepackage{steinmetz}
\usepackage{tikz}
\usepackage{circuitikz}
\usepackage{verbatim}
\usepackage{tfrupee}
\usepackage[breaklinks=true]{hyperref}
\usepackage{graphicx}
\usepackage{tkz-euclide}

\usetikzlibrary{calc,math}
\usepackage{listings}
    \usepackage{color}                                            %%
    \usepackage{array}                                            %%
    \usepackage{longtable}                                        %%
    \usepackage{calc}                                             %%
    \usepackage{multirow}                                         %%
    \usepackage{hhline}                                           %%
    \usepackage{ifthen}                                           %%
    \usepackage{lscape}     
\usepackage{multicol}
\usepackage{chngcntr}

\DeclareMathOperator*{\Res}{Res}

\renewcommand\thesection{\arabic{section}}
\renewcommand\thesubsection{\thesection.\arabic{subsection}}
\renewcommand\thesubsubsection{\thesubsection.\arabic{subsubsection}}

\renewcommand\thesectiondis{\arabic{section}}
\renewcommand\thesubsectiondis{\thesectiondis.\arabic{subsection}}
\renewcommand\thesubsubsectiondis{\thesubsectiondis.\arabic{subsubsection}}


\hyphenation{op-tical net-works semi-conduc-tor}
\def\inputGnumericTable{}                                 %%

\lstset{
%language=C,
frame=single, 
breaklines=true,
columns=fullflexible
}
\begin{document}


\newtheorem{theorem}{Theorem}[section]
\newtheorem{problem}{Problem}
\newtheorem{proposition}{Proposition}[section]
\newtheorem{lemma}{Lemma}[section]
\newtheorem{corollary}[theorem]{Corollary}
\newtheorem{example}{Example}[section]
\newtheorem{definition}[problem]{Definition}

\newcommand{\BEQA}{\begin{eqnarray}}
\newcommand{\EEQA}{\end{eqnarray}}
\newcommand{\define}{\stackrel{\triangle}{=}}
\bibliographystyle{IEEEtran}
\raggedbottom
\setlength{\parindent}{0pt}
\providecommand{\mbf}{\mathbf}
\providecommand{\pr}[1]{\ensuremath{\Pr\left(#1\right)}}
\providecommand{\qfunc}[1]{\ensuremath{Q\left(#1\right)}}
\providecommand{\sbrak}[1]{\ensuremath{{}\left[#1\right]}}
\providecommand{\lsbrak}[1]{\ensuremath{{}\left[#1\right.}}
\providecommand{\rsbrak}[1]{\ensuremath{{}\left.#1\right]}}
\providecommand{\brak}[1]{\ensuremath{\left(#1\right)}}
\providecommand{\lbrak}[1]{\ensuremath{\left(#1\right.}}
\providecommand{\rbrak}[1]{\ensuremath{\left.#1\right)}}
\providecommand{\cbrak}[1]{\ensuremath{\left\{#1\right\}}}
\providecommand{\lcbrak}[1]{\ensuremath{\left\{#1\right.}}
\providecommand{\rcbrak}[1]{\ensuremath{\left.#1\right\}}}
\theoremstyle{remark}
\newtheorem{rem}{Remark}
\newcommand{\sgn}{\mathop{\mathrm{sgn}}}
\providecommand{\abs}[1]{\left\vert#1\right\vert}
\providecommand{\res}[1]{\Res\displaylimits_{#1}} 
\providecommand{\norm}[1]{\left\lVert#1\right\rVert}
%\providecommand{\norm}[1]{\lVert#1\rVert}
\providecommand{\mtx}[1]{\mathbf{#1}}
\providecommand{\mean}[1]{E\left[ #1 \right]}
\providecommand{\fourier}{\overset{\mathcal{F}}{ \rightleftharpoons}}
%\providecommand{\hilbert}{\overset{\mathcal{H}}{ \rightleftharpoons}}
\providecommand{\system}{\overset{\mathcal{H}}{ \longleftrightarrow}}
 %\newcommand{\solution}[2]{\textbf{Solution:}{#1}}
\newcommand{\solution}{\noindent \textbf{Solution: }}
\newcommand{\cosec}{\,\text{cosec}\,}
\providecommand{\dec}[2]{\ensuremath{\overset{#1}{\underset{#2}{\gtrless}}}}
\newcommand{\myvec}[1]{\ensuremath{\begin{pmatrix}#1\end{pmatrix}}}
\newcommand{\mydet}[1]{\ensuremath{\begin{vmatrix}#1\end{vmatrix}}}
\numberwithin{equation}{subsection}
\makeatletter
\@addtoreset{figure}{problem}
\makeatother
\let\StandardTheFigure\thefigure
\let\vec\mathbf
\renewcommand{\thefigure}{\theproblem}
\def\putbox#1#2#3{\makebox[0in][l]{\makeb
ox[#1][l]{}\raisebox{\baselineskip}[0in][0in]{\raisebox{#2}[0in][0in]{#3}}}}
     \def\rightbox#1{\makebox[0in][r]{#1}}
     \def\centbox#1{\makebox[0in]{#1}}
     \def\topbox#1{\raisebox{-\baselineskip}[0in][0in]{#1}}
     \def\midbox#1{\raisebox{-0.5\baselineskip}[0in][0in]{#1}}
\vspace{3cm}
\title{Assignment 6 - Q54, June 2018}
\author{Tarandeep Singh}
\maketitle
\newpage
\bigskip
\renewcommand{\thefigure}{\theenumi}
\renewcommand{\thetable}{\theenumi}
Github Link
\begin{lstlisting}
https://github.com/Tarandeep97/AI5030
\end{lstlisting}
\section{Problem}
(Q54, June 2018) Let $X_{1},X_{2},X_{3}..X_{n}$ be i.i.d. uniform $(\theta_{1},\theta_{2})$ random variables, where $\theta_{1}<\theta_{2}$ are unknown parameters. Which of the following is an ancillary statistic?
\begin{enumerate}
\item $\frac{X_{(k)}}{X_{(n)}}$ for any $k<n$
\item $\frac{X_{(n)}-X_{(k)}}{X_{(n)}}$ for any $k<n$
\item $\frac{X_{(k)}}{X_{(n)}-X_{(k)}}$ for any $k<n$
\item $\frac{X_{(k)}-X_{(1)}}{X_{(n)}-X_{(k)}}$ for any $k<n$ where $1<k<n$
\end{enumerate}


\section{Solution}
\textbf{Definitions}\\
Statistic: A statistic is any function of the observations in a sample.\\
Ancillary statistic: A statistic S(X) is an ancillary statistic if its distribution does not depend on $\theta$.\\
Order Statistics: If $X_{1},X_{2},X_{3}..X_{n}$ are observations of a random sample of size n from a continuous distribution, we arrange the random variables as:
$X^\prime_{1}<X^\prime_{2}<X^\prime_{3}...<X^\prime_{n}$\\
where $X^\prime_{k}$ is the kth order statistic i.e. kth smallest among all observations.
\\
Clearly, $X_{i}\sim U(\theta_{1},\theta_{2})$ is not an ancillary statistic as it depends on $\theta_{1}$ and $\theta_{2}$ i.e. its pdf and cdf is given by,
\begin{align}
     f_{X_{i}}(x) = \frac{1}{\theta_{2}-\theta_{1}}
\end{align}
\begin{align}
     F_{X_{i}}(x) =P(X_{i}<x)= \frac{x-\theta_{1}}{\theta_{2}-\theta_{1}}
\end{align}
for $\theta_{1}\leq x_{i}\leq\theta_{2}$\\
These  Random variables$X_{i}\sim U(\theta_{1},\theta_{2})$ can be converted to Standard Uniform Random Variables as follows,\\
As we know,
\begin{align}
    \theta_{1}\leq x_{i}\leq\theta_{2}
\end{align}
\begin{align}
    0\leq x_{i}-\theta_{1}\leq\theta_{2}-\theta_{1}
\end{align}
\begin{align}
    0\leq \frac{x_{i}-\theta_{1}}{\theta_{2}-\theta_{1}}\leq 1
\end{align}
Considering, 
\begin{align}
    y_{i}=\frac{x_{i}-\theta_{1}}{\theta_{2}-\theta_{1}}
\end{align}
Hence, given distribution of $X_{i}\sim U(\theta_{1},\theta_{2})$ can be converted to $Y_{i}\sim U(0,1)$ which is independent of given parameters.\\
This also implies that each of these $Y_{i}$ and their combination will be an ancillary statistic as its pdf and cdf are as follows,
\begin{align}
     f_{Y_{i}}(x) = 1
\end{align}
\begin{align}
     F_{Y_{i}}(x) = P(Y\leq x)
\end{align}
\begin{align}
     F_{Y_{i}}(x) = P\left(\frac{x_{i}-\theta_{1}}{\theta_{2}-\theta_{1}}\leq x\right)
\end{align}
\begin{align}
     F_{Y_{i}}(x) = P\left[x_{i}\leq x(\theta_{2}-\theta_{1})+\theta_{1}\right]
\end{align}
Using 2.0.2,
\begin{align}
     F_{Y_{i}}(x) = x
\end{align}

\\
Option 1 can be rewritten as, 
\begin{align}
    \frac{X_{(k)}}{X_{(n)}} =\frac{X_{(k)}-\theta_{1}+\theta_{1}}{X_{(n)}-\theta_{1}+\theta_{1}}
\end{align}
\begin{align}
    =\frac{\frac{X_{(k)}-\theta_{1}}{\theta_{2}-\theta_{1}}+\frac{\theta_{1}}{\theta_{2}-\theta_{1}}}{\frac{X_{(n)}-\theta_{1}}{\theta_{2}-\theta_{1}}+\frac{\theta_{1}}{\theta_{2}-\theta_{1}}}
\end{align}
\begin{align}
    =\frac{Y_{k}+\frac{\theta_{1}}{\theta_{2}-\theta_{1}}}{Y_{n}+\frac{\theta_{1}}{\theta_{2}-\theta_{1}}}
\end{align}
Option 2 can be rewritten as, 
\begin{align}
    \frac{X_{(n)}-X_{(k)}}{X_{(n)}} 
    =\frac{X_{(n)}-\theta_{1}-(X_{(k)}-\theta_{1})}{X_{(n)}-\theta_{1}+\theta_{1}}
\end{align}
\begin{align}
    =\frac{\frac{X_{(n)}-\theta_{1}}{\theta_{2}-\theta_{1}}+\frac{X_{(k)}-\theta_{1}}{\theta_{2}-\theta_{1}}}{\frac{X_{(n)}-\theta_{1}}{\theta_{2}-\theta_{1}}+\frac{\theta_{1}}{\theta_{2}-\theta_{1}}}
\end{align}
\begin{align}
    =\frac{Y_{n}+Y_{k}}{Y_{n}+\frac{\theta_{1}}{\theta_{2}-\theta_{1}}}
\end{align}
Option 3 can be rewritten as, 
\begin{align}
    \frac{X_{(k)}}{X_{(n)}-X_{(k)}} 
    =\frac{X_{(k)}-\theta_{1}+\theta_{1}}{X_{(n)}-\theta_{1}-(X_{(k)}-\theta_{1})}
\end{align}
\begin{align}
    =\frac{\frac{X_{(k)}-\theta_{1}}{\theta_{2}-\theta_{1}}+\frac{\theta_{1}}{\theta_{2}-\theta_{1}}}{\frac{X_{(n)}-\theta_{1}}{\theta_{2}-\theta_{1}}+\frac{X_{(k)}-\theta_{1}}{\theta_{2}-\theta_{1}}}
\end{align}
\begin{align}
    =\frac{Y_{k}+\frac{\theta_{1}}{\theta_{2}-\theta_{1}}}{Y_{n}+Y_{k}}
\end{align}
Option 4 can be rewritten as, 
\begin{align}
    \frac{X_{(k)}-X_{(1)}}{X_{(n)}-X_{(k)}} 
    =\frac{X_{(k)}-\theta_{1}-(X_{(1)}-\theta_{1})}{X_{(n)}-\theta_{1}-(X_{(k)}-\theta_{1})}
\end{align}
\begin{align}
    =\frac{\frac{X_{(k)}-\theta_{1}}{\theta_{2}-\theta_{1}}+\frac{X_{(1)}-\theta_{1}}{\theta_{2}-\theta_{1}}}{\frac{X_{(n)}-\theta_{1}}{\theta_{2}-\theta_{1}}+\frac{X_{(k)}-\theta_{1}}{\theta_{2}-\theta_{1}}}
\end{align}
\begin{align}
    =\frac{Y_{k}+Y_{1}}{Y_{n}+Y_{k}}
\end{align}
Since, only option 4 does not depend on $\theta_{1}$ and $\theta_{2}$, it an ancillary statistic.
\end{document}
