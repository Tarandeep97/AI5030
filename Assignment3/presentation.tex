\documentclass{beamer}
\usetheme{Madrid}
\usecolortheme{beaver}
\usepackage{amsmath}
\usepackage{verbatim}

\title{Probability Axioms}
\subtitle{Assignment 3 - Q49, Dec 2018}
\author{Tarandeep Singh}
\institute{IIT Hyderabad}
\date{\today}

\begin{document}

\begin{frame}
\titlepage
\end{frame}
\begin{comment}
\section{Introduction}


\end{comment}
\section{The Problem}
\subsection{Problem Statement}

\begin{frame}
\frametitle{Problem Statement}
Let $X \geq 0$ be a random variable on  $(\Omega,{\mathcal {F}}, P)$ with $\mathbb{E}(X)=1$. Let $A\in{\mathcal {F}}$ be an event with $ 0 < P(A) < 1$. Which of the following defines another probability measure on $(\Omega,{\mathcal {F}})$?
\begin{enumerate}
\item $Q(B) = P(A\cap B)$     $\forall B \in {\mathcal {F}}$
\item $Q(B) = P(A\cup B)$     $\forall B \in {\mathcal {F}}$
\item $Q(B) = \mathbb{E}(XI_{B})$     $\forall B \in {\mathcal {F}}$ 
\item $Q(B) = \begin{cases} 
      P(A/B) & P(B)> 0 \\
      0 & P(B) = 0 
   \end{cases}$

\end{enumerate}
\end{frame}



\subsection{Problem Explanation}
\begin{frame}
\frametitle{Properties of Probability Measure}
A Probability measure on $(\Omega,{\mathcal {F}})$ is a function $P:{\mathcal {F}} \rightarrow [0,1]$ satisfying:
\begin{enumerate}
\item $P(\Omega)=1, P(\phi)=0$
\item If $A_{1},A_{2},A_{3}... \in {\mathcal {F}}$  is a collection of disjoint members in ${\mathcal {F}}$, then  \\
 \[P\left(\bigcup_{i=1}^{\infty} A_{i}\right)= \sum_{i=1}^{\infty} P(A_{i}) \]  
\end{enumerate}
\end{frame}

\section{Solution}
\begin{frame}
\frametitle{Solution}
Option 1, if $B= \Omega$, then
\begin{align}
    Q(\Omega) = P(A \cap \Omega) = P(A) \neq 1
\end{align}

Option 2, if $B= \phi$, then
\begin{align}
    Q(\phi) = P(A \cup \phi) = P(A) \neq 0
\end{align}
Option 4, if $B= \Omega$, then
\begin{align}
    Q(\Omega) = P(A/\Omega) = \frac{P(A \cap \Omega)}{P(\Omega)}
    = P(A)\neq 1
\end{align}
All these options doesn't satisfy the properties of Probability measure. Hence, option 3 is right option.
\end{frame}

\section{Solution}
\begin{frame}
\frametitle{Solution}
Option 3, The indicator function $I_{B}$ defines a Bernoulli random variable. Then $\forall x \in \Omega$ 
\begin{align}
I_{B}(x) = \begin{cases}
1 & \text{ if $x\in B$},\\
0 & \text{ if $x\notin B$},
\end{cases}
\end{align}
if $B=\phi$, 
\begin{align}
    Q(\phi) = \mathbb{E}(XI_{\phi}) = \sum_{x\in\Omega} \sum_{x\in\Omega} x.I_{B}(x)P((X=x).(I_{B}(x)))
\end{align}
Here, $\forall x \in \Omega$, $I_{B}(x)=0$, as $B=\phi$. \\
if $B=\Omega$, 
\begin{align}
    Q(\Omega) = \mathbb{E}(XI_{\Omega}) =\sum_{x\in\Omega} \sum_{x\in\Omega} x.I_{B}(x)P((X=x).(I_{B}(x))) \\= \mathbb{E}(X)=1
\end{align}
Here, $\forall x \in \Omega$, $I_{B}(x)=1$, as $B=\Omega$. \\
\end{frame}

\section{Solution}
\begin{frame}
\frametitle{Solution}
Checking for Countable additivity of Q(property 2),
\begin{align}
    Q\left(\bigcup_{i=1}^{\infty} A_{i}\right) = \mathbb{E}(XI_{\bigcup_{i=1}^{\infty} A_{i}\right)})
\end{align}
Using Property of Countable additivity on indicator functions expectation can be re-written as below,
\begin{align}
    =\mathbb{E}\left(X\sum_{i=1}^{\infty}I_{A_{i}}\right) =\mathbb{E}\left(\sum_{i=1}^{\infty}XI_{A_{i}}\right)
    =\sum_{i=1}^{\infty}\mathbb{E}\left(XI_{A_{i}}\right)
\end{align}
\begin{align}
     \sum_{i=1}^{\infty}Q(A_{i})
\end{align}
\end{frame}


\end{document}
